% 默认页面大小 4:3
\documentclass[10pt]{ctexbeamer}
% 页面大小 16:10
% \documentclass[10pt, aspectratio=1610]{ctexbeamer}
% 页面大小 16:9
% \documentclass[10pt, aspectratio=169]{ctexbeamer}
% 页面大小 14:9
% \documentclass[10pt, aspectratio=149]{ctexbeamer}
% 页面大小 1.41:1
% \documentclass[10pt, aspectratio=141]{ctexbeamer}
% 页面大小 5:4
% \documentclass[10pt, aspectratio=54]{ctexbeamer}
% 页面大小 3:2
% \documentclass[10pt, aspectratio=32]{ctexbeamer}

\usetheme[logo=UCAS, sublogo=naoc]{ucas}
% logo 的选项: CAS, UCAS
% sublogo 的选项: AMSS, AMSS2018, UCAS

% 引入参考文献列表的 .bib 文件, 使用 GB/T 7714-2015 的文献著录规则.
\usepackage[backend=biber, style=gb7714-2015]{biblatex}
\addbibresource{ref.bib}

\title[UCAS Beamer (\LaTeX{})]{Transformer与时间序列分析}
% \subtitle[非官方]{非官方的模版}
\author[H. Zeng]{\href{mailto:haozeng1210@gmail.com}{曾滈}}
\institute[NAOC]{中国科学院国家天文台}
\date[\today]{\today, 北京怀柔}
\subject{展示主题}
\keywords{展示, 关键词}

\begin{document}

\begin{frame}[plain]
  \maketitle
\end{frame}

\begin{frame}[t]
  \frametitle{目录}
  \tableofcontents
\end{frame}

\section{第一章:Transformer与时间序列分析}

\begin{frame}{序列模型架构}
  \frametitle{挑战与解决方法}
  \begin{itemize}
    \item 挑战:转换器模型在处理序列数据方面极具潜力,特别是在处理天文时间序列数据时。然而,由于数据中的测量误差和不规律采样,需要适应这些模型来处理天文数据的独特挑战。
    \item 解决方法:Astronet通过高斯过程插值和规律间隔的重采样技术有效解决了不规律采样问题,而ATAT模型则在天体分类中,通过融合光变曲线的红移等元信息到转换器架构中,并利用可学习的傅立叶系数来处理不规律采样,避免了高斯过程回归成本高昂的问题。
  \end{itemize}
\end{frame}


\section{第二章:预训练范式与编码器改进}

\begin{frame}{预训练范式及改进部分}
  \begin{itemize}
    \item \textbf{天文时间序列分析的预训练范式}
          \begin{itemize}
            \item 预训练范式提高了模型性能,成为该领域重要工具。
          \end{itemize}
    \item \textbf{TimeModAttn模型的编码器阶段}
          \begin{itemize}
            \item 结合自编码器架构和注意力机制,优化状态空间模型的重构。
          \end{itemize}
    \item \textbf{位置编码的创新}
          \begin{itemize}
            \item 可训练的位置编码在变星分类中表现出色。
          \end{itemize}
  \end{itemize}
\end{frame}


\section{第三章:天文图像的应用}

\begin{frame}{Transformer Vesus CNN}
  \begin{itemize}
    \item \textbf{Transformer在天文图像中的应用}
          \begin{itemize}
            \item 处理Subaru望远镜的重叠图像,实现识别、解混合和分类任务顺序执行。
          \end{itemize}
    \item \textbf{CNN与计算成本的权衡}
          \begin{itemize}
            \item 在计算资源有限时,CNN处理小规模数据集效率高,仍然有其价值。
          \end{itemize}
  \end{itemize}
\end{frame}

\begin{frame}{模型的可解释性与CBS-GPT的应用}
  \begin{itemize}
    \item \textbf{增强模型可解释性}
          \begin{itemize}
            \item 使用注意力图和工具来可视化输入数据中的关键部分,帮助理解决策过程。(但好像仅仅是输入数据的阶段)
          \end{itemize}
    \item \textbf{CBS-GPT模型介绍}
          \begin{itemize}
            \item 针对波形复杂性问题,提高紧凑双星系统波形的预测能力和解释性。
          \end{itemize}
  \end{itemize}
\end{frame}




\begin{frame}[noframenumbering, allowframebreaks, t]
  \frametitle{参考文献}
  \nocite{*}% 打印未引用,但已列入 .bib 文件内的文献
  \printbibliography%
\end{frame}

\begin{frame}[plain]
  \vfill
  \centerline{\Huge 谢谢}
  \vfill
\end{frame}

%  
\end{document}
