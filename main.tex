% 默认页面大小 4:3
\documentclass[10pt]{ctexbeamer}
% 页面大小 16:10
% \documentclass[10pt, aspectratio=1610]{ctexbeamer}
% 页面大小 16:9
% \documentclass[10pt, aspectratio=169]{ctexbeamer}
% 页面大小 14:9
% \documentclass[10pt, aspectratio=149]{ctexbeamer}
% 页面大小 1.41:1
% \documentclass[10pt, aspectratio=141]{ctexbeamer}
% 页面大小 5:4
% \documentclass[10pt, aspectratio=54]{ctexbeamer}
% 页面大小 3:2
% \documentclass[10pt, aspectratio=32]{ctexbeamer}

\usetheme[logo=UCAS, sublogo=naoc]{ucas}
% logo 的选项: CAS, UCAS
% sublogo 的选项: AMSS, AMSS2018, UCAS

% 引入参考文献列表的 .bib 文件, 使用 GB/T 7714-2015 的文献著录规则.
\usepackage[backend=biber, style=gb7714-2015]{biblatex}
\addbibresource{ref.bib}

\title[UCAS Beamer (\LaTeX{})]{使用集成强化方式处理不平衡数据}
% \subtitle[非官方]{非官方的模版}
\author[H. Zeng]{\href{mailto:haozeng1210@gmail.com}{曾滈}}
\institute[NAOC]{中国科学院国家天文台}
\date[\today]{\today, 北京怀柔}
\subject{展示主题}
\keywords{展示, 关键词}

\begin{document}

\begin{frame}[plain]
  \maketitle
\end{frame}

\begin{frame}[t]
  \frametitle{目录}
  \tableofcontents
\end{frame}


\section{第1章 问题与现有方法}\label{sec:1}

\begin{frame}[t]
  \frametitle{分类周期变星存在的问题}
  不同类别周期变星的样本量存在高度不平衡。
\end{frame}

\begin{frame}[t]
  \frametitle{现有的解决方法}
  \begin{itemize}
    \item 传统机器学习算法:SMOTE、Self-paced Ensemble等
    \item 数据生成:添加噪声、高斯过程、深度生成模型
  \end{itemize}
\end{frame}

\section{第2章 本文方法与结果}\label{sec:2}

\begin{frame}[t]
  \frametitle{数据层面}
  利用高斯过程生成小样本量类别的合成光变曲线。
\end{frame}

\begin{frame}[t]
  \frametitle{模型层面}
  \begin{itemize}
    \item RNN-based神经网络
    \item RNN + CNN混合结构神经网络
  \end{itemize}
\end{frame}

\begin{frame}[t]
  \frametitle{训练优化}
  采用bagging-like方式组织数据增强和模型训练,减轻过拟合。
\end{frame}

\begin{frame}[t]
  \frametitle{取得的主要结果}
  \begin{itemize}
    \item 在CRTS数据集上Macro F1得分达到0.75。
    \item 总体精度达到86.2\%。
  \end{itemize}
\end{frame}

\section{第3章 与现有方法的对比}\label{sec:3}

\begin{frame}[t]
  \frametitle{与现有方法的不同}
  \begin{itemize}
    \item 使用高斯过程而不是SMOTE生成合成数据。
    \item 尝试了不同的神经网络架构。
    \item 采用了bagging-like的模型集成方法。
  \end{itemize}
\end{frame}

\section{第4章 展望}\label{sec:4}



\begin{frame}[t]
  \frametitle{未来展望}
  \begin{itemize}
    \item 继续优化小样本类别分类。
    \item 探索更高效的模型和训练方式。
    \item 在其他数据上验证所提出的方法。
  \end{itemize}
\end{frame}




\begin{frame}[noframenumbering, allowframebreaks, t]
  \frametitle{参考文献}
  \nocite{*}% 打印未引用,但已列入 .bib 文件内的文献
  \printbibliography%
\end{frame}

\begin{frame}[plain]
  \vfill
  \centerline{\Huge 谢谢}
  \vfill
\end{frame}

%  
\end{document}
