\documentclass[sans]{beamer}
\usepackage{ctex}

% \usetheme[label = UCAS, logo = UCAS]{ucas}
\usetheme[]{ucas}

\title[主题缩写]{主题}
\subtitle[副标题缩写]{副标题}
\author[G.W.\, CH]{\href{mailto:icgw@outlook.com}{纯果味}}
\institute[AMSS, CAS]{中国科学院数学与系统科学研究院}
\date{\today}
\subject{展示主题}
\keywords{展示,关键词}

\begin{document}
\begin{frame}[plain]
	\maketitle
\end{frame}

\begin{frame}{目录}
	\tableofcontents
\end{frame}

\section[第 1 章缩写标题]{第 1 章主标题}
\subsection[第 1 节缩写标题]{第 1 节主标题}

\begin{frame}{幻灯片标题}{幻灯片副标题}
	平凡格式,\structure{浅灰格式},\alert{强调格式}
	\begin{itemize}
		\item 第一级文本内容
		\item 若该行文本内容十分长长长长长长长长长,则会被强制换行,
		      这里也可以包含\alert{需要强调的文本}
		      \begin{itemize}
			      \item 第二级文本内容
			            \begin{itemize}
				            \item 第三级文本内容
			            \end{itemize}
			            \alert{\item 第二级强调的文本内容}
		      \end{itemize}
	\end{itemize}
	\begin{enumerate}
		\item 带序号的文本内容
		      \begin{enumerate}
			      \item 第二级文本内容且包含数学公式
			            \[ E = mc^2 \]
		      \end{enumerate}
	\end{enumerate}
\end{frame}

\subsection[第 2 节缩写标题]{第 2 节主标题}

\begin{frame}{文本区块}
	将文本放入区块内
	\begin{block}{普通区块}
		引用 \cite{icgw} 提供的模板
	\end{block}
	\begin{exampleblock}{示例区块}
		\red{红色}
	\end{exampleblock}
	\begin{alertblock}{强调区块}
		\green{绿色}
	\end{alertblock}
	给文本加脚注\footnote{这个脚注附有链接 \url{http://icgw.me}}
\end{frame}

% \begin{frame}{图像}
% 	\begin{figure}
% 		\includegraphics[width=.5\textwidth, height=.5\textheight, keepaspectratio]{cow-black.mps}
% 		\caption{荷斯坦黑白花牛}
% 	\end{figure}
% \end{frame}

\subsection[第 3 节缩写标题]{第 3 节主标题}

\begin{frame}{表格}
	\begin{table}
		\begin{tabular}{llc}
			名称     & 姓氏     & 出生年份 \\ \midrule
			阿尔伯特 & 爱因斯坦 & 1879     \\
			玛丽     & 居里     & 1867     \\
			托马斯   & 爱迪生   & 1847     \\
		\end{tabular}
		\caption{19世纪伟大的科学家}
	\end{table}
\end{frame}

\makeatletter
\begin{frame}{自定义字体大小}
	\begin{center}
		\begin{tabular}{ll}
			\Huge  大大大大大字号 & \Huge \structure{\f@size pt}         \\
			\huge  大大大大字号   & \huge \structure{\f@size pt}         \\
			\LARGE 大大大字号     & \LARGE \structure{\f@size pt}        \\
			\Large 大大字号       & \Large \structure{\f@size pt}        \\
			\large 大字号的{\songti 宋体}          & \large \structure{\f@size pt}        \\
			\normalsize 正常字号的{\heiti 黑体}    & \normalsize \structure{\f@size pt}   \\
			\small 小字号的{\fangsong 仿宋}        & \small \structure{\f@size pt}        \\
			\footnotesize 小小字号的{\kaishu 楷书} & \footnotesize \structure{\f@size pt} \\
			\scriptsize 小小小字号 & \scriptsize \structure{\f@size pt}   \\
			\tiny 小小小小字号     & \tiny \structure{\f@size pt}
		\end{tabular}
	\end{center}
\end{frame}
\makeatother

\begin{frame}[allowframebreaks, plain]
	\frametitle<presentation>{参考文献}
	\begin{thebibliography}{10}
		\setbeamertemplate{bibliography item}[online]
		\bibitem[纯果味]{icgw}% 网络资源
			纯果味
			\newblock {\em 中国科学院大学} Beamer (LaTeX) 模板(非官方版本)
			\newblock \url{https://github.com/icgw/ucas-beamer}

		\setbeamertemplate{bibliography item}[book]
		\bibitem[Knuth, 1997]{knuth1997art}% 书籍
			Knuth, Donald Ervin% 作者
			\newblock The art of computer programming: sorting and searching% 书名
			\newblock Vol. 3. {\em Pearson Education}, 1997.% 出版社

		\setbeamertemplate{bibliography item}[article]
		\bibitem[Jain, 1999]{jain1999data}% 文章
			Jain, Anil K and Murty, M Narasimha and Flynn, Patrick. J% 作者
			\newblock Data clustering: a review% 题目
			\newblock {\em ACM computing surveys (CSUR)}, 31(3): 264-323, 1999.% 期刊
		\setbeamertemplate{bibliography item}[triangle]
		\bibitem[项目名]{symbol} 项目内容

		\setbeamertemplate{bibliography item}[text]
		\bibitem[标题]{text} 文本内容

		\setbeamertemplate{bibliography item}[book]
		\bibitem[Knuth, 1997]{tantau2004user}% 书籍
			Tantau, Till and Wright, J and Miletic, V% 作者
			\newblock User Guide to the Beamer Class% 书名
			\newblock {\em Version}, 2004.% 出版社
	\end{thebibliography}
\end{frame}

\begin{frame}[plain]
	\vfill
	\centerline{\Large 谢谢!}
	\vfill\vfill
\end{frame}

\end{document}