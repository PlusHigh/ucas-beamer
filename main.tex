% 默认页面大小 4:3
\documentclass[10pt]{beamer}
% 页面大小 16:10
% \documentclass[10pt, aspectratio=1610]{beamer}
% 页面大小 16:9
% \documentclass[10pt, aspectratio=169]{beamer}
% 页面大小 14:9
% \documentclass[10pt, aspectratio=149]{beamer}
% 页面大小 1.41:1
% \documentclass[10pt, aspectratio=141]{beamer}
% 页面大小 5:4
% \documentclass[10pt, aspectratio=54]{beamer}
% 页面大小 3:2
% \documentclass[10pt, aspectratio=32]{beamer}

\usetheme[logo=UCAS, sublogo=AMSS]{ucas}
% logo 的选项: CAS, UCAS
% sublogo 的选项: AMSS, AMSS2018, UCAS

\usepackage[backend=biber, citestyle=authoryear]{biblatex}

% 具有字形变化的字母编码工具包,若不需要,可注释或删除,
% \usepackage[T1]{fontenc}
% 引入 [T1]{fontenc] 工具包在低版本 TeX 编译环境可能因为某些字符,如`~' 导致编译失败,以下命令可解决问题
% \DeclareTextCommand{\nobreakspace}{T1}{\leavevmode\nobreak\ }

% 用于超文本链接的工具包
\usepackage{hyperref}

% 引入参考文献列表的 .bib 文件
\addbibresource{ref.bib}

\title[UCAS Beamer (\LaTeX{})]{中国科学院大学 Beamer (\LaTeX{})}
\subtitle[非官方]{非官方的模版}
\author[G.\,Chen]{\href{mailto:icgw@outlook.com}{陈国威}}
\institute[AMSS, CAS]{中国科学院数学与系统科学研究院}
\date[\today]{\today, 中国北京}
\subject{展示主题}
\keywords{展示,关键词}

\begin{document}

\begin{frame}[plain]
  \maketitle
\end{frame}

\begin{frame}[t]
  \frametitle{目录}
  \tableofcontents
\end{frame}

\section[第 1 章缩写标题]{第 1 章主标题}\label{sec:1}
\subsection[第 1 节缩写标题]{第 1 节主标题}\label{subsec:1-1}

\begin{frame}[t]
  \frametitle{幻灯片标题}
  \framesubtitle{幻灯片副标题}
  平凡格式,\structure{浅灰格式},\alert{强调格式}
  \begin{itemize}
    \item 第一级文本内容
    \item 若该行文本内容十分长长长长长长长长长,则会被强制换行,
      这里也可以包含\alert{需要强调的文本}
      \begin{itemize}
        \item 第二级文本内容
          \begin{itemize}
            \item 第三级文本内容
          \end{itemize}
          \alert{\item 第二级强调的文本内容}
      \end{itemize}
  \end{itemize}
  \begin{enumerate}
    \item 带序号的文本内容
      \begin{enumerate}
        \item 第二级文本内容且包含数学公式
          \[\int^{\infty}_{-\infty}e^{-x^2}dx = \sqrt{\pi}\]
      \end{enumerate}
  \end{enumerate}
\end{frame}

\subsection[第 2 节缩写标题]{第 2 节主标题}\label{subsec:1-2}

\begin{frame}[t]
  \frametitle{文本区块}
  将文本放入区块内
  \begin{block}{普通区块}
    引用 \autocite{guowei2019ucasbeamer} 提供的模板
  \end{block}
  \begin{exampleblock}{示例区块}
    \red{红色}
  \end{exampleblock}
  \begin{alertblock}{强调区块}
    \green{绿色}
  \end{alertblock}
  给文本加脚注\footnote{这个脚注附有链接 \url{https://github.com/icgw/ucas-beamer}}
\end{frame}

% \begin{frame}[t]
%   \frametitle{图像}
%   \begin{figure}
%     \includegraphics[width=.5\textwidth, height=.5\textheight, keepaspectratio]{cow-black.mps}
%     \caption{荷斯坦黑白花牛}
%   \end{figure}
% \end{frame}

\subsection[第 3 节缩写标题]{第 3 节主标题}\label{subsec:1-3}

\begin{frame}[t]
  \frametitle{表格}
  \begin{table}
    \begin{tabular}{lcl}\toprule
      姓名   & 出生年份 & 母校 (本科)  \\ \midrule
      陶哲轩 & 1975     & 弗林德斯大学 \\
      张益唐 & 1955     & 北京大学     \\
      丘成桐 & 1949     & 香港中文大学 \\ \bottomrule
    \end{tabular}
    \caption{二十一世纪的数学家}
  \end{table}
\end{frame}

\makeatletter
\begin{frame}[t]
  \frametitle{自定义字体大小}
  \begin{center}
    \begin{tabular}{ll}
      \Huge  $\backslash$Huge                               & \Huge \structure{24.88 pt}     \\
      \huge  $\backslash$huge                               & \huge \structure{20.74 pt}     \\
      \LARGE $\backslash$LARGE                              & \LARGE \structure{17.28 pt}    \\
      \Large $\backslash$Large                              & \Large \structure{14.4 pt}     \\
      \large $\backslash$large{\songti{宋体}} (小四)        & \large \structure{12 pt}       \\
      \normalsize $\backslash$normalsize{\heiti{黑体}}      & \normalsize \structure{10 pt}  \\
      \small $\backslash$small{\fangsong{仿宋}}             & \small \structure{9 pt}        \\
      \footnotesize $\backslash$footnotesize{\kaishu{楷书}} & \footnotesize \structure{8 pt} \\
      \scriptsize $\backslash$scriptsize                    & \scriptsize \structure{7 pt}   \\
      \tiny $\backslash$ting (八号)                         & \tiny \structure{5 pt}
    \end{tabular}
  \end{center}
\end{frame}
\makeatother

\begin{frame}[noframenumbering, allowframebreaks, t]
  \frametitle{参考文献}
  \nocite{*}% 打印未引用,但已列入 .bib 文件内的文献
  \printbibliography%
\end{frame}

\begin{frame}[plain]
  \vfill
  \centerline{\Huge 谢谢!}
  \vfill
\end{frame}

\end{document}
