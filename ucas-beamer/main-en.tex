\documentclass[
]{beamer}

\usepackage[english]{babel}
\usepackage{inputenc}
\usepackage[T1]{fontenc}
\usepackage{booktabs}
\usetheme[
  workplace=ucas,
]{UCAS}

\title[Short Presentation Title]{Full Presentation Title}
\subtitle[Short Presentation Subtitle]{Full Presentation Subtitle}
\author[G.W.\, CH]{Chen Guowei \\ icgw@outlook.com}
\institute[UCAS]{University of Chinese Academy of Science}
\date{\today}
\subject{Presentation Subject}
\keywords{the, presentation, keywords}

\begin{document}
\begin{frame}[plain]
\maketitle
\end{frame}

\begin{frame}{Table of Contents}
  \tableofcontents
\end{frame}

\section[Short Section 1 Name]{Full Section 1 Name}
\subsection[Short Subsection 1 Name]{Full Subsection 1 Name}

\begin{frame}{Frame Title}{Frame Subtitle}
plain text, \structure{page structure}, \alert{emphasis}
\begin{itemize}
  \item a single-line bullet list item
  \item a bullet list item that is quite long (in order to force a line break),
    which also contains \alert{emphasized text}
  \begin{itemize}
    \item a second-level list item
    \begin{itemize}
      \item a third-level list item
    \end{itemize}
    \alert{\item an emphasized second-level list item}
  \end{itemize}
\end{itemize}
\begin{enumerate}
  \item a numbered list item
  \begin{enumerate}
    \item a second-level list item containing a math expression
      \[ E = mc^2 \]
  \end{enumerate}
\end{enumerate}
\end{frame}

\subsection[Short Subsection 2 Name]{Full Subsection 2 Name}

\begin{frame}{Text Blocks}
text above a block
\begin{block}{Block}
  text
\end{block}
\begin{exampleblock}{Example Block}
  text
\end{exampleblock}
\begin{alertblock}{Emphasized Block}
  text
\end{alertblock}
text below a block\footnote{a footnote with an \url{http://icgw.me}}
\end{frame}

\begin{frame}{Figures}
\begin{figure}
  \includegraphics[width=.5\textwidth,height=.5\textheight,keepaspectratio]{cow-black.mps}
  \caption{A Holstein Friesian cow}
\end{figure}
\end{frame}

\subsection[Short Subsection 3 Name]{Full Subsection 3 Name}

\begin{frame}{Tables}
\begin{table}
  \begin{tabular}{llc}
    First Name & Surname & Year of Birth \\ \midrule
    Albert & Einstein & 1879 \\
    Marie & Curie & 1867 \\
    Thomas & Edison & 1847 \\
  \end{tabular}
  \caption{The great minds of the 19th century}
\end{table}
\end{frame}

\makeatletter
\begin{frame}{Automatic Optical Scaling}
\begin{center}
\begin{tabular}{ll}
\Huge Huge & \Huge \structure{\f@size pt} \\
\huge huge & \huge \structure{\f@size pt}  \\
\LARGE LARGE & \LARGE \structure{\f@size pt}  \\
\Large Large & \Large \structure{\f@size pt}  \\
\large large & \large \structure{\f@size pt}  \\
\normalsize normalsize & \normalsize \structure{\f@size pt}  \\[-0.95pt]
\small small & \small \structure{\f@size pt}  \\[-1.95pt]
\footnotesize footnotesize & \footnotesize \structure{\f@size pt} \\[-2.95pt]
\scriptsize scriptsize & \scriptsize \structure{\f@size pt}  \\[-4.95pt]
\tiny tiny & \tiny \structure{\f@size pt}
\end{tabular}
\end{center}
\end{frame}
\makeatother

\begin{frame}[plain]
\vfill
\centerline{Thank you for your attention!}
\vfill\vfill
\end{frame}

\end{document}
