\documentclass[
]{beamer}

\usepackage{ctex}
\usepackage{inputenc}
\usepackage[T1]{fontenc}
\usepackage{booktabs}
\usetheme[
  workplace=ucas,
]{UCAS}

\title[主题缩写]{主题}
\subtitle[副标题缩写]{副标题}
\author[G.W.\, CH]{纯果味 \\ icgw@outlook.com}
\institute[UCAS]{中国科学院大学}
\date{\today}
\subject{展示主题}
\keywords{展示,关键词}

\begin{document}
\begin{frame}[plain]
\maketitle
\end{frame}

\begin{frame}{目录}
\tableofcontents
\end{frame}

\section[第 1 章缩写标题]{第 1 章全称标题}
\subsection[第 1 节缩写标题]{第 1 节全称标题}

\begin{frame}{幻灯片标题}{幻灯片副标题}
平凡格式,\structure{列表格式},\alert{强调格式}
\begin{itemize}
  \item 第一级文本内容
  \item 若该行文本内容十分长长长长长长长长长,则会被强制换行,
    这里也可以包含 \alert{需要强调的文本}
  \begin{itemize}
    \item 第二级文本内容
    \begin{itemize}
      \item 第三级文本内容
    \end{itemize}
    \alert{\item 第二级强调的文本内容}
  \end{itemize}
\end{itemize}
\begin{enumerate}
  \item 带序号的文本内容
  \begin{enumerate}
    \item 第二级文本内容且包含数学公式
      \[ E = mc^2 \]
  \end{enumerate}
\end{enumerate}
\end{frame}

\subsection[第 2 节缩写标题]{第 2 节全称标题}

\begin{frame}{文本区块}
将文本放入区块内
\begin{block}{普通区块}
  文本内容
\end{block}
\begin{exampleblock}{示例区块}
  文本内容
\end{exampleblock}
\begin{alertblock}{强调区块}
  文本内容
\end{alertblock}
给文本加脚注\footnote{这个脚注附有链接 \url{http://icgw.me}}
\end{frame}

\begin{frame}{图像}
\begin{figure}
  \includegraphics[width=.5\textwidth,height=.5\textheight,keepaspectratio]{cow-black.mps}
  \caption{荷斯坦黑白花牛}
\end{figure}
\end{frame}

\subsection[第 3 节缩写标题]{第 3 节全称标题}

\begin{frame}{表格}
\begin{table}
  \begin{tabular}{llc}
    名称 & 姓氏 & 出生年份 \\ \midrule
    阿尔伯特 & 爱因斯坦 & 1879 \\
    玛丽 & 居里 & 1867 \\
    托马斯 & 爱迪生 & 1847 \\
  \end{tabular}
  \caption{19世纪伟大的科学家}
\end{table}
\end{frame}

\makeatletter
\begin{frame}{自定义字体大小}
\begin{center}
\begin{tabular}{ll}
\Huge \f@family & \Huge \structure{\f@size pt} \\
\huge \f@family & \huge \structure{\f@size pt}  \\
\LARGE \f@family & \LARGE \structure{\f@size pt}  \\
\Large \f@family & \Large \structure{\f@size pt}  \\
\large \f@family & \large \structure{\f@size pt}  \\
\normalsize \f@family & \normalsize \structure{\f@size pt}  \\[-0.95pt]
\small \f@family & \small \structure{\f@size pt}  \\[-1.95pt]
\footnotesize \f@family & \footnotesize \structure{\f@size pt} \\[-2.95pt]
\scriptsize \f@family & \scriptsize \structure{\f@size pt}  \\[-4.95pt]
\tiny \f@family & \tiny \structure{\f@size pt}
\end{tabular}
\end{center}
\end{frame}
\makeatother

\begin{frame}[plain]
\vfill
\centerline{谢谢!}
\vfill\vfill
\end{frame}

\end{document}
